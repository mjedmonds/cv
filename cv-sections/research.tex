%----------------------------------------------------------------------------------------
%	SECTION TITLE
%----------------------------------------------------------------------------------------

\cvsection{Research}

%----------------------------------------------------------------------------------------
%	SECTION CONTENT
%----------------------------------------------------------------------------------------

\begin{cventries}

%------------------------------------------------

% STICK TO "accomplished X measured by Y by doing Z" for the descriptions

\cventry
{Graduate Student Researcher; Center for Vision, Cognition, Learning, and Autonomy (VCLA)}
{Causal Transfer Learning}
{Los Angeles, CA}
{Sep 2017 – Present}
{
\begin{cvitems}
\item Examining how causal knowledge can be incorporated into reinforcement learning to enable better knowledge transfer across task and environment domains.
\item Decomposed human causal learning into two components: a bottom-up associative learning scheme and a top-down structure learning scheme.
\item Studied how humans perform in causal transfer tasks and compared performance against state-of-the-art reinforcement learning algorithms.
\end{cvitems}
}

%------------------------------------------------

\cventry
{Graduate Student Researcher; Center for Vision, Cognition, Learning, and Autonomy (VCLA)}
{Imitation Learning using Tactile Feedback}
{Los Angeles, CA}
{Sep 2015 – Sep 2017}
{
\begin{cvitems}
\item Transferred visually latent causal changes from a human demonstrator to a robot using a tactile glove, an And-Or graph, and neural networks.
\item The manipulation policy uses the And-Or graph to encode long-term temporal constraints and uses haptic feedback to incorporate real-time sensor data.
\item Deployed robot localization on a ROS-based Baxter robot combining SLAM (using RGB-D and LIDAR), wheel odometry, and IMU data through Kalman filtering.
\end{cvitems}
}

%------------------------------------------------

\cventry
{Undergraduate Researcher; Air Force Research Lab (AFRL)}
{Declarative Memory Acceleration}
{Dayton, OH}
{May 2014 – Sep 2015}
{
\begin{cvitems}
\item Accelerated the declarative memory module of AFRL's CECEP cognitive architecture (based on \href{http://act-r.psy.cmu.edu/}{ACT-R}) by leveraging the parallelization of CUDA, yielding a 100x speedup over the fastest existing implementation.
\item Utilized CUDA, thread pools, and IPC to achieve the speedup.
\end{cvitems}
}

%------------------------------------------------

\end{cventries}